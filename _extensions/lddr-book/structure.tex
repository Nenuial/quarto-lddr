%%%%%%%%%%%%%%%%%%%%%%%%%%%%%%%%%%%%%%%%%
% The Legrand Orange Book
% Structural Definitions File
% Version 2.0 (9/2/15)
%
% Original author:
% Mathias Legrand (legrand.mathias@gmail.com) with modifications by:
% Vel (vel@latextemplates.com)
%
% This file has been downloaded from:
% http://www.LaTeXTemplates.com
%
% License:
% CC BY-NC-SA 3.0 (http://creativecommons.org/licenses/by-nc-sa/3.0/)
%
%%%%%%%%%%%%%%%%%%%%%%%%%%%%%%%%%%%%%%%%%

%----------------------------------------------------------------------------------------
%	VARIOUS REQUIRED PACKAGES AND CONFIGURATIONS
%----------------------------------------------------------------------------------------
\usepackage{import}

\usepackage[top=3cm,bottom=3cm,left=3cm,right=3cm,headsep=10pt,a4paper]{geometry} % Page margins

%\usepackage{amsmath,amsfonts,amssymb,amsthm} % For math equations, theorems, symbols, etc
%\usepackage{mathabx}
%\let\ulcorner\relax
%\let\urcorner\relax
%\let\llcorner\relax
%\let\lrcorner\relax
%\usepackage{wasysym}

\usepackage{index}
\usepackage{imakeidx}
\usepackage{datatool}

\usepackage{tabularx}
\usepackage{longtable}
\usepackage{array,colortbl,multirow,multicol,adjustbox,booktabs}
\newcolumntype{C}{>{\centering\arraybackslash}X}
\newcolumntype{R}{>{\raggedleft\arraybackslash}X}
\newcolumntype{L}{>{\raggedright\arraybackslash}X}

\usepackage{tabto}

\usepackage{float}
\floatstyle{plain}
\newfloat{flbox}{thb}{flb}
\floatname{flbox}{}

\usepackage{iflang}

\usepackage{pdflscape}

\usepackage{newfloat}

\usepackage{graphicx} % Required for including pictures
\usepackage{tikz} % Required for drawing custom shapes
\usetikzlibrary{shapes.geometric}
\usetikzlibrary{arrows.meta,arrows}
\usetikzlibrary{backgrounds}
\usetikzlibrary{bending}
\usetikzlibrary{calc}
\usetikzlibrary{decorations}
\usetikzlibrary{intersections}
\usetikzlibrary{angles,quotes}
\usetikzlibrary{decorations.text}
\usetikzlibrary{positioning}
\usetikzlibrary{fadings}


\usepackage[space]{grffile} % For spaces in paths
\usepackage{etoolbox} % For spaces in paths
\makeatletter % For spaces in paths
\patchcmd\Gread@eps{\@inputcheck#1 }{\@inputcheck"#1"\relax}{}{}
\makeatother


\usepackage{lipsum} % Inserts dummy text
\usepackage{verbatim} % Commenting large sections

\usepackage{enumitem} % Customize lists
\setlist{nolistsep} % Reduce spacing between bullet points and numbered lists
\providecommand{\tightlist}{%
  \setlength{\itemsep}{0pt}\setlength{\parskip}{0pt}}

\usepackage{siunitx}
\sisetup{inter-unit-product = \ensuremath{\cdot}}%
\sisetup{exponent-product = \cdot}%

\usepackage{xcolor} % Required for specifying colors by name
\definecolor{maincolor}{HTML}{0F2D61} % Define the color used for highlighting throughout the book
\definecolor{ddrblue}{RGB}{36, 32, 75} % Define the DDR Color

\usepackage{pgf}
\usepackage{pgfplots}
\usepackage{pgfplotstable}

\usepackage{caption}


%----------------------------------------------------------------------------------------
%	GRAPHIC CONFIGURATIONS
%----------------------------------------------------------------------------------------

\pgfplotsset{%
        compat=1.9,
        blank onypyramid axis style/.style={%
            width=0.38*\textwidth,
            height=0.39*\textheight,
            scale only axis,
            grid=both,
    		grid style={line width=.1pt, draw=gray!10},
    		major grid style={line width=.2pt,draw=gray!50},
            xmin=0,
            ymin=-0.5,
            ymax=100,
            y dir=reverse,
            enlarge x limits={value=1,upper},
            enlarge y limits={value=0.006,upper},
            xbar,
            axis x line=left,
            xtick align=outside,
            bar width=1,
            allow reversal of rel axis cs=false,
        },
        onypyramid axis style/.style={%
            blank onypyramid axis style,
            ytick=\empty,
            axis line style={-},
        },
    }

%----------------------------------------------------------------------------------------
%	FONTS AND FORMATTING
%----------------------------------------------------------------------------------------
\usepackage[french=quotes]{csquotes}
%\usepackage{avant} % Use the Avantgarde font for headings
%\usepackage{times} % Use the Times font for headings
%\usepackage{ulem}
%\normalem % Italic for emphasis
%\usepackage{mathptmx} % Use the Adobe Times Roman as the default text font together with math symbols from the Sym­bol, Chancery and Com­puter Modern fonts

%\usepackage{textcomp} % Allows for direct degree symbols
\usepackage{newunicodechar}
\newunicodechar{°}{\textdegree} % Define ° as \textdegree

%\usepackage{indentfirst}

%\setlength{\parskip}{0.5em}
\setlength{\parindent}{0.3cm}

% dashed itemize
\def\labelitemi{---}
\def\labelitemii{--}

% roman numerals
\makeatletter
\newcommand*{\romannum}[1]{\expandafter\@slowromancap\romannumeral #1@}
\makeatother

% Translation macro
\newcommand{\trad}[2]{\IfLanguageName{english}{#1}{#2}}

%----------------------------------------------------------------------------------------
%	FLOATS AND FORMATTING BEHAVIOUR
%----------------------------------------------------------------------------------------

\renewcommand{\floatpagefraction}{.8}%

\widowpenalty 5000
\clubpenalty 5000


%----------------------------------------------------------------------------------------
%	BIBLIOGRAPHY AND INDEX
%----------------------------------------------------------------------------------------

\usepackage{calc} % For simpler calculation - used for spacing the index letter headings correctly
%\usepackage{makeidx} % Required to make an index
\makeindex[intoc] % Tells LaTeX to create the files required for indexing

%----------------------------------------------------------------------------------------
%	MAIN TABLE OF CONTENTS
%----------------------------------------------------------------------------------------

\usepackage{titletoc} % Required for manipulating the table of contents

\contentsmargin{0cm} % Removes the default margin

% Part text styling
\titlecontents{part}[0cm]
{\addvspace{20pt}\centering\large\bfseries}
{}
{}
{}

% Chapter text styling
\titlecontents{chapter}[1.25cm] % Indentation
{\addvspace{12pt}\large\sffamily\bfseries} % Spacing and font options for chapters
{\color{maincolor!60}\contentslabel[\Large\thecontentslabel]{1.25cm}\color{maincolor}} % Chapter number
{\color{maincolor}}
{\color{maincolor!60}\normalsize\;\titlerule*[.5pc]{.}\;\thecontentspage} % Page number

% Section text styling
\titlecontents{section}[1.25cm] % Indentation
{\addvspace{3pt}\sffamily\bfseries} % Spacing and font options for sections
{\contentslabel[\thecontentslabel]{1.25cm}} % Section number
{}
{\color{black}\;\titlerule*[.5pc]{.}\;\thecontentspage} % Page number
[]

% Subsection text styling
\titlecontents{subsection}[1.25cm] % Indentation
{\addvspace{1pt}\sffamily\small} % Spacing and font options for subsections
{\contentslabel[\thecontentslabel]{1.25cm}} % Subsection number
{}
{\ \titlerule*[.5pc]{.}\;\thecontentspage} % Page number
[]

% List of figures
\titlecontents{figure}[0em]
{\addvspace{-5pt}\sffamily}
{\thecontentslabel\hspace*{1em}}
{}
{\ \titlerule*[.5pc]{.}\;\thecontentspage}
[]

% List of tables
\titlecontents{table}[0em]
{\addvspace{-5pt}\sffamily}
{\thecontentslabel\hspace*{1em}}
{}
{\ \titlerule*[.5pc]{.}\;\thecontentspage}
[]

%----------------------------------------------------------------------------------------
%	MINI TABLE OF CONTENTS IN PART HEADS
%----------------------------------------------------------------------------------------

% Chapter text styling
\titlecontents{lchapter}[0em] % Indenting
{\addvspace{15pt}\large\sffamily\bfseries} % Spacing and font options for chapters
{\color{maincolor}\contentslabel[\Large\thecontentslabel]{1.25cm}\color{maincolor}} % Chapter number
{}
{\color{maincolor}\normalsize\sffamily\bfseries\;\titlerule*[.5pc]{.}\;\thecontentspage} % Page number

% Section text styling
\titlecontents{lsection}[0em] % Indenting
{\sffamily\small} % Spacing and font options for sections
{\contentslabel[\thecontentslabel]{1.25cm}} % Section number
{}
{}

% Subsection text styling
\titlecontents{lsubsection}[.5em] % Indentation
{\normalfont\footnotesize\sffamily} % Font settings
{}
{}
{}

%----------------------------------------------------------------------------------------
%	PAGE HEADERS
%----------------------------------------------------------------------------------------

\usepackage{fancyhdr} % Required for header and footer configuration

\pagestyle{fancy}
\renewcommand{\chaptermark}[1]{\markboth{\sffamily\normalsize\bfseries\chaptername\ \thechapter.\ #1}{}} % Chapter text font settings
\renewcommand{\sectionmark}[1]{\markright{\sffamily\normalsize\thesection\hspace{5pt}#1}{}} % Section text font settings
\fancyhf{} \fancyhead[LE,RO]{\sffamily\normalsize\thepage} % Font setting for the page number in the header
\fancyhead[LO]{\rightmark} % Print the nearest section name on the left side of odd pages
\fancyhead[RE]{\leftmark} % Print the current chapter name on the right side of even pages
\renewcommand{\headrulewidth}{0.5pt} % Width of the rule under the header
\addtolength{\headheight}{2.5pt} % Increase the spacing around the header slightly
\renewcommand{\footrulewidth}{0pt} % Removes the rule in the footer
\fancypagestyle{plain}{\fancyhead{}\renewcommand{\headrulewidth}{0pt}} % Style for when a plain pagestyle is specified

% Removes the header from odd empty pages at the end of chapters
\makeatletter
\renewcommand{\cleardoublepage}{
\clearpage\ifodd\c@page\else
\hbox{}
\vspace*{\fill}
\thispagestyle{empty}
\newpage
\fi}


%----------------------------------------------------------------------------------------
%	SECTION NUMBERING IN THE MARGIN
%----------------------------------------------------------------------------------------

\makeatletter
\renewcommand{\@seccntformat}[1]{\llap{\textcolor{maincolor}{\csname the#1\endcsname}\hspace{1em}}}
\renewcommand{\section}{\@startsection{section}{1}{\z@}
{-4ex \@plus -1ex \@minus -.4ex}
{1ex \@plus.2ex }
{\normalfont\large\sffamily\bfseries}}
\renewcommand{\subsection}{\@startsection {subsection}{2}{\z@}
{-3ex \@plus -0.1ex \@minus -.4ex}
{0.5ex \@plus.2ex }
{\normalfont\sffamily\bfseries}}
\renewcommand{\subsubsection}{\@startsection {subsubsection}{3}{\z@}
{-2ex \@plus -0.1ex \@minus -.2ex}
{.2ex \@plus.2ex }
{\normalfont\small\sffamily\bfseries}}
\renewcommand\paragraph{\@startsection{paragraph}{4}{\z@}
{-2ex \@plus-.2ex \@minus .2ex}
{.1ex}
{\normalfont\small\sffamily\bfseries}}

%----------------------------------------------------------------------------------------
%	PART HEADINGS
%----------------------------------------------------------------------------------------

% A switch to conditionally color background, implemented by  Christian Hupfer
\newif\ifpartbackground
\partbackgroundtrue

% numbered part in the table of contents
\newcommand{\@mypartnumtocformat}[2]{%
\setlength\fboxsep{0pt}%
\noindent\colorbox{maincolor!20}{\strut\parbox[c][.7cm]{\ecart}{\color{maincolor!70}\Large\sffamily\bfseries\centering#1}}\hskip\esp\colorbox{maincolor!40}{\strut\parbox[c][.7cm]{\linewidth-\ecart-\esp}{\Large\sffamily\centering#2}}}%
%%%%%%%%%%%%%%%%%%%%%%%%%%%%%%%%%%
% unnumbered part in the table of contents
\newcommand{\@myparttocformat}[1]{%
\setlength\fboxsep{0pt}%
\noindent\colorbox{maincolor!40}{\strut\parbox[c][.7cm]{\linewidth}{\Large\sffamily\centering#1}}}%
%%%%%%%%%%%%%%%%%%%%%%%%%%%%%%%%%%
\newlength\esp
\setlength\esp{4pt}
\newlength\ecart
\setlength\ecart{1.2cm-\esp}
\newcommand{\thepartimage}{}%
\newcommand{\partimage}[1]{\renewcommand{\thepartimage}{#1}}%
\def\@part[#1]#2{%
\ifnum \c@secnumdepth >-2\relax%
\refstepcounter{part}%
\addcontentsline{toc}{part}{\texorpdfstring{\protect\@mypartnumtocformat{\thepart}{#1}}{\partname~\thepart\ ---\ #1}}
\else%
\addcontentsline{toc}{part}{\texorpdfstring{\protect\@myparttocformat{#1}}{#1}}%
\fi%
\startcontents%
\markboth{}{}%
{\thispagestyle{empty}%
\begin{tikzpicture}[remember picture,overlay]%
\node at (current page.north west){\begin{tikzpicture}[remember picture,overlay]%
\ifpartbackground\fill[maincolor!20](0cm,0cm) rectangle (\paperwidth,-\paperheight);\fi
\node[anchor=north] at (4cm,-3.25cm){\color{maincolor!40}\fontsize{220}{100}\addfontfeature{LetterSpace=-8}\bfseries\@Roman\c@part};
\node[anchor=south east] at (\paperwidth-1cm,-\paperheight+1cm){\parbox[t][][t]{8.5cm}{
\printcontents{l}{0}{\setcounter{tocdepth}{1}}%
}};
\node[anchor=north east] at (\paperwidth-1.5cm,-3.25cm){\parbox[t][][t]{15cm}{\strut\raggedleft\ifpartbackground\color{white}\else\color{maincolor}\fi\fontsize{30}{30}\sffamily\bfseries#2}};
\end{tikzpicture}};
\end{tikzpicture}}%
\@endpart}
\def\@spart#1{%
\startcontents%
\phantomsection
{\thispagestyle{empty}%
\begin{tikzpicture}[remember picture,overlay]%
\node at (current page.north west){\begin{tikzpicture}[remember picture,overlay]%
\ifusechapterimage\fill[maincolor!20](0cm,0cm) rectangle (\paperwidth,-\paperheight);
\node[anchor=north east] at (\paperwidth-1.5cm,-3.25cm){\parbox[t][][t]{15cm}{\strut\raggedleft\color{white}\fontsize{30}{30}\sffamily\bfseries#1}};
\end{tikzpicture}};
\end{tikzpicture}}
\addcontentsline{toc}{part}{\texorpdfstring{%
\setlength\fboxsep{0pt}%
\noindent\protect\colorbox{maincolor!40}{\strut\protect\parbox[c][.7cm]{\linewidth}{\Large\sffamily\protect\centering #1\quad\mbox{}}}}{#1}}%
\@endpart}
\def\@endpart{\vfil\newpage
\if@twoside
\if@openright
\null
\thispagestyle{empty}%
\newpage
\fi
\fi
\if@tempswa
\twocolumn
\fi}

%----------------------------------------------------------------------------------------
%	CHAPTER HEADINGS
%----------------------------------------------------------------------------------------

% A switch to conditionally include a picture, implemented by  Christian Hupfer
\newif\ifusechapterimage
\usechapterimagefalse
\newcommand{\thechapterimage}{}%
\newcommand{\chapterimage}[1]{\ifusechapterimage\renewcommand{\thechapterimage}{#1}\fi}%
\def\@makechapterhead#1{%
{\parindent \z@ \raggedright \normalfont
\ifnum \c@secnumdepth >\m@ne
\if@mainmatter
\begin{tikzpicture}[remember picture,overlay]
\node at (current page.north west)
{\begin{tikzpicture}[remember picture,overlay]
\node[anchor=north west,inner sep=0pt] at (0,0) {\ifusechapterimage\includegraphics[width=\paperwidth]{\thechapterimage}\fi};
\draw[anchor=west] (\Gm@lmargin,-9cm) node [line width=2pt,rounded corners=15pt,draw=maincolor,fill=white,fill opacity=0.7,inner sep=15pt]{\strut\makebox[22cm]{}};
\draw[anchor=west] (\Gm@lmargin+.3cm,-9cm) node {\huge\sffamily\bfseries\color{black}\thechapter. #1\strut};
\end{tikzpicture}};
\end{tikzpicture}
\else
\begin{tikzpicture}[remember picture,overlay]
\node at (current page.north west)
{\begin{tikzpicture}[remember picture,overlay]
\node[anchor=north west,inner sep=0pt] at (0,0) {\ifusechapterimage\includegraphics[width=\paperwidth]{\thechapterimage}\fi};
\draw[anchor=west] (\Gm@lmargin,-9cm) node [line width=2pt,rounded corners=15pt,draw=maincolor,fill=white,fill opacity=0.5,inner sep=15pt]{\strut\makebox[22cm]{}};
\draw[anchor=west] (\Gm@lmargin+.3cm,-9cm) node {\huge\sffamily\bfseries\color{black}#1\strut};
\end{tikzpicture}};
\end{tikzpicture}
\fi\fi\par\vspace*{270\p@}}}

%-------------------------------------------

\def\@makeschapterhead#1{%
\begin{tikzpicture}[remember picture,overlay]
\node at (current page.north west)
{\begin{tikzpicture}[remember picture,overlay]
\node[anchor=north west,inner sep=0pt] at (0,0) {\ifusechapterimage\includegraphics[width=\paperwidth]{\thechapterimage}\fi};
\draw[anchor=west] (\Gm@lmargin,-9cm) node [line width=2pt,rounded corners=15pt,draw=maincolor,fill=white,fill opacity=0.5,inner sep=15pt]{\strut\makebox[22cm]{}};
\draw[anchor=west] (\Gm@lmargin+.3cm,-9cm) node {\huge\sffamily\bfseries\color{black}#1\strut};
\end{tikzpicture}};
\end{tikzpicture}
\par\vspace*{270\p@}}
\makeatother

%----------------------------------------------------------------------------------------
%	PGFPLOT ENGINEERING NOTATION
%----------------------------------------------------------------------------------------

\makeatletter

\newif\ifpgfplots@scaled@x@ticks@engineering
\pgfplots@scaled@x@ticks@engineeringfalse
\newif\ifpgfplots@scaled@y@ticks@engineering
\pgfplots@scaled@y@ticks@engineeringfalse
\newif\ifpgfplots@scaled@z@ticks@engineering
\pgfplots@scaled@z@ticks@engineeringfalse

\pgfplotsset{
    scaled x ticks/engineering/.code=
        \pgfplots@scaled@x@ticks@engineeringtrue,
    scaled y ticks/engineering/.code=
        \pgfplots@scaled@y@ticks@engineeringtrue,
    scaled z ticks/engineering/.code=
        \pgfplots@scaled@y@ticks@engineeringtrue,
%    scaled ticks=engineering  % Uncomment this line if you want "engineering" to be on by default
}

\def\pgfplots@init@scaled@tick@for#1{%
    \global\def\pgfplots@glob@TMPa{0}%
    \expandafter\pgfplotslistcheckempty\csname pgfplots@prepared@tick@positions@major@#1\endcsname
    \ifpgfplotslistempty
        % we have no tick labels. Omit the tick scale label as well!
    \else
    \begingroup
    \ifcase\csname pgfplots@scaled@ticks@#1@choice\endcsname\relax
    % CASE 0 : scaled #1 ticks=false: do nothing here.
    \or
        % CASE 1 : scaled #1 ticks=true:
        %--------------------------------
        % the \pgfplots@xmin@unscaled@as@float  is set just before the data
        % scale transformation is initialised.
        %
        % The variables are empty if there is no datascale transformation.
        \expandafter\let\expandafter\pgfplots@cur@min@unscaled\csname pgfplots@#1min@unscaled@as@float\endcsname
        \expandafter\let\expandafter\pgfplots@cur@max@unscaled\csname pgfplots@#1max@unscaled@as@float\endcsname
        %
        \ifx\pgfplots@cur@min@unscaled\pgfutil@empty
            \edef\pgfplots@loc@TMPa{\csname pgfplots@#1min\endcsname}%
            \expandafter\pgfmathfloatparsenumber\expandafter{\pgfplots@loc@TMPa}%
            \let\pgfplots@cur@min@unscaled=\pgfmathresult
            \edef\pgfplots@loc@TMPa{\csname pgfplots@#1max\endcsname}%
            \expandafter\pgfmathfloatparsenumber\expandafter{\pgfplots@loc@TMPa}%
            \let\pgfplots@cur@max@unscaled=\pgfmathresult
        \fi
        %
        \expandafter\pgfmathfloat@decompose@E\pgfplots@cur@min@unscaled\relax\pgfmathfloat@a@E
        \expandafter\pgfmathfloat@decompose@E\pgfplots@cur@max@unscaled\relax\pgfmathfloat@b@E
        \pgfplots@init@scaled@tick@normalize@exponents
        \ifnum\pgfmathfloat@b@E<\pgfmathfloat@a@E
            \pgfmathfloat@b@E=\pgfmathfloat@a@E
        \fi
        \xdef\pgfplots@glob@TMPa{\pgfplots@scale@ticks@above@exponent}%
        \ifnum\pgfplots@glob@TMPa<\pgfmathfloat@b@E
            % ok, scale it:
            \expandafter\ifx % Check whether we're using engineering notation (restricting exponents to multiples of three)
                \csname ifpgfplots@scaled@#1@ticks@engineering\expandafter\endcsname
                \csname iftrue\endcsname
                    \divide\pgfmathfloat@b@E by 3
                    \multiply\pgfmathfloat@b@E by 3
            \fi
            \multiply\pgfmathfloat@b@E by-1
            \xdef\pgfplots@glob@TMPa{\the\pgfmathfloat@b@E}%
        \else
            \xdef\pgfplots@glob@TMPa{\pgfplots@scale@ticks@below@exponent}%
            \ifnum\pgfplots@glob@TMPa>\pgfmathfloat@b@E
                % ok, scale it:
                \expandafter\ifx % Check whether we're using engineering notation (restricting exponents to multiples of three)
                    \csname ifpgfplots@scaled@#1@ticks@engineering\expandafter\endcsname
                    \csname iftrue\endcsname
                        \advance\pgfmathfloat@b@E by -2
                        \divide\pgfmathfloat@b@E by 3
                        \multiply\pgfmathfloat@b@E by 3
                \fi
                \multiply\pgfmathfloat@b@E by-1
                \xdef\pgfplots@glob@TMPa{\the\pgfmathfloat@b@E}%
            \else
                % no scaling necessary:
                \xdef\pgfplots@glob@TMPa{0}%
            \fi
        \fi
    \or
        % CASE 2 : scaled #1 ticks=base 10:
        %--------------------------------
        \c@pgf@counta=\csname pgfplots@scaled@ticks@#1@arg\endcsname\relax
        %\multiply\c@pgf@counta by-1
        \xdef\pgfplots@glob@TMPa{\the\c@pgf@counta}%
    \or
        % CASE 3 : scaled #1 ticks=real:
        %--------------------------------
        \pgfmathfloatparsenumber{\csname pgfplots@scaled@ticks@#1@arg\endcsname}%
        \global\let\pgfplots@glob@TMPa=\pgfmathresult
    \or
        % CASE 4 : scaled #1 ticks=manual:
        \expandafter\global\expandafter\let\expandafter\pgfplots@glob@TMPa\csname pgfplots@scaled@ticks@#1@arg\endcsname
    \fi
    \endgroup
    \fi
    \expandafter\let\csname pgfplots@tick@scale@#1\endcsname=\pgfplots@glob@TMPa%
}
\makeatother

%----------------------------------------------------------------------------------------
%	HYPERLINKS IN THE DOCUMENTS
%----------------------------------------------------------------------------------------

\usepackage{zref-xr,zref-user}
\usepackage{hyperref}
\hypersetup{hidelinks,backref=true,pagebackref=true,hyperindex=true,colorlinks=false,breaklinks=true,urlcolor= maincolor,bookmarks=true,bookmarksopen=false,pdftitle={Géographie DF},pdfauthor={Pascal Burkhard}}
\usepackage{bookmark}
\bookmarksetup{
open,
numbered,
addtohook={%
\ifnum\bookmarkget{level}=0 % chapter
\bookmarksetup{bold}%
\fi
\ifnum\bookmarkget{level}=-1 % part
\bookmarksetup{color=maincolor,bold}%
\fi
}
}
